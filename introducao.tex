\section{Introdução}
\label{s.introducao}

\begin{center}
\emph{Machine Learning} (mais teórico) x \emph{Pattern Recognition} x \emph{Artificial Intelligence} (área maior, lógica fuzzy)
\end{center}

Podemos considerar o \emph{Machine Learning} e o \emph{Pattern Recognition} como uma inteligência artificial aplicada, isto é, redes neurais, SVM, \emph{Deep Learning}.

\begin{center}
Imagem $\longrightarrow$ Extração de características $\longrightarrow$ Espaço de características	
\end{center}

O conjunto de dados é composto por um conjunto de treinamento, utilizado para aprender e determinar a função separadora do espaço (parâmetros). E também é composto por um conjunto de teste, utilizado em novas amostras e para definir o grupo das mesmas.

\begin{center}
\emph{Machine Learning} $\longrightarrow$ supervisionado (dados rotulados) ou não supervisionado
\end{center}

\subsection{Exemplos de Técnicas}
\label{ss.tecnicas}

\begin{itemize}
	\item SVM - support vector machine: parte do princípio que os dados podem ser separados. O problema se dá quando aumentamos o número de parâmetros, trazendo um maior custo computacional;

	\item Reconhecimento de padrões: o problema é dividido em classificação e regressão (encontrar o fitting, ou seja, a melhor função que modela o problema).	
\end{itemize}
